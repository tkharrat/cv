% Created 2018-04-23 Mon 10:28
% Intended LaTeX compiler: pdflatex
\documentclass[a4paper,twoside,11pt]{article}
                              \usepackage[T1]{fontenc}
\usepackage{libertine}
\renewcommand*\oldstylenums[1]{{\fontfamily{fxlj}\selectfont #1}}
\usepackage{lmodern}
\usepackage[margin=1.0in]{geometry}
\usepackage{rotating}
\usepackage{setspace,amsmath,amsfonts,amssymb,bm}
\usepackage{graphicx}
\usepackage[usenames,dvipsnames]{xcolor}
\definecolor{Red}{rgb}{0.5,0,0}
\definecolor{NavyBlue}{rgb}{0.1,0.1,0.45}
\definecolor{MidnightBlue}{rgb}{0.1,0.1,0.65}
\usepackage[pdftex,hyperfootnotes=true,plainpages=false,pdfpagelabels,hypertexnames=true,naturalnames,pdfproducer={Latex},pdfcreator={pdflatex},bookmarks,bookmarksnumbered,colorlinks,linkcolor=MidnightBlue,citecolor=NavyBlue,filecolor=black,urlcolor=Red,breaklinks=true]{hyperref}
\usepackage{authblk}
\usepackage{xspace,helvet}
\usepackage{moreverb}
\usepackage{url, booktabs}
\usepackage{cleveref}
\usepackage[pdftex]{lscape}
\usepackage{fullpage}
\usepackage{booktabs}
\usepackage{multirow}
\usepackage{rotating}
\usepackage{pifont}
\usepackage{setspace}
\usepackage{threeparttable}
\usepackage{tabulary}
\usepackage[toc,page]{appendix}
\usepackage{pbox}
\usepackage[font=small]{caption}
\newcommand{\rom}[1]{\uppercase\expandafter{\romannumeral #1\relax}}
\newcommand{\E}{\mathsf{E}}
\newcommand{\VAR}{\mathsf{VAR}}
\newcommand{\COV}{\mathsf{COV}}
\newcommand{\Prob}{\mathsf{P}}
\newcommand{\RNum}[1]{\uppercase\expandafter{\romannumeral #1\relax}}
\newcommand{\dee}{\,\mbox{d}}
\newcommand{\naive}{na\"{\i}ve }
\newcommand{\eg}{e.g.\xspace}
\newcommand{\ie}{i.e.\xspace}
\newcommand{\pdf}{pdf.\xspace}
\newcommand{\etc}{etc.\@\xspace}
\newcommand{\PhD}{Ph.D.\xspace}
\newcommand{\MSc}{M.Sc.\xspace}
\newcommand{\BA}{B.A.\xspace}
\newcommand{\R}{\texttt{R}\xspace}
\usepackage{paralist}
\let\itemize\compactitem
\let\description\compactdesc
\let\enumerate\compactenum
\let\enumerate\inparaenum
\renewenvironment{enumerate}{\begin{inparaenum}[(i)]}{\end{inparaenum}}
\renewenvironment{enumerate}{\begin{inparaenum}[(a)]}{\end{inparaenum}}
\usepackage[round]{natbib}
\author[1, 2]{Tarak Kharrat}
\affil[1]{University of Liverpool, London Campus, UK.}
\affil[2]{Kickdex Limited, Lodon, UK.}
\newcommand{\countr}{\texttt{Countr}\xspace}
\date{\today}
\title{my Technical Toolbox\\\medskip
\large An Overview}
\hypersetup{
 pdfauthor={tarak},
 pdftitle={my Technical Toolbox},
 pdfkeywords={},
 pdfsubject={},
 pdfcreator={Emacs 25.3.1 (Org mode 9.0.7)}, 
 pdflang={English}}
\begin{document}

\maketitle
\begin{abstract}
The main motivation of this document is to give the reader an overview of my
technical toolbox and the projects I have been involved in. It is meant to
complement my (short) resume.
\end{abstract}

\section{Research Activity}
\label{sec:org0d483c0}
I am currently a (part-time) research fellow at the University of Liverpool
where I undertake research projects that fall under the umbrella of
\emph{'Computational Statistics'}. In this context, together with colleagues, we
developed methods and software to solve problems in:
\begin{itemize}
\item \textbf{Counting Processes:} We developed a new family of counting processes that
generalise the standard Poisson and negative binomial models. These models
have the nice property to allow fitting over as well as under-dispersed
data. The theoretical properties are discussed in \citet{baker2017event}. An
\texttt{R} package \citet{Rcore} has been published on \href{https://CRAN.R-project.org/package=Countr}{CRAN} and described in a
dedicated paper \citep{kharrat2018jss}.

\item \textbf{Time Series:} We developed new methodologies to fit non-Gaussian non-linear
state space models. The methods are implemented in the \texttt{R} package \texttt{GKF} (for
internal use only).

\item \textbf{Statistical Distributions:} We created new estimation techniques for the 4
parameters of Stable law distributions. These methods converge faster and
still enjoy the normal asymptotic properties. This work is discussed in
\cite{kharrat2015jss}.
\item \textbf{Survival Analysis:} I have been involved in some research with time to event
data. In particular, I have a large experience with competing risks models and
their multi-state generalisation.
\end{itemize}

\section{Sports Analytics}
\label{sec:orgfb52d45}
As co-head of R\&D at Kickdex Limited, a start-up specialised in predictive
modelling in football, I developed many tools and metrics to analyse, in a
purely quantitative way, the game of football. Also the applied side of my
research is dedicated to football. In fact, some chapters in my PhD thesis 
\citep{TarakPhd} discuss some models suggested mainly to forecast the match
score grid. In the rest of this section, I detail some of the projects I worked
on in this area:

\begin{itemize}
\item I created a unique database (from scratch) by web-scrapping the
internet. These database contains event-by-event data (similar to
opta F24), video game players information (EA Sports FIFA and Konami PES),
betting prices (both pre-match and in play for different markets) and
historical injury record. The database is updated weekly and is stored on
the \href{https://www.mongodb.com/cloud/atlas}{mongo-Atlas} cloud.

\item My primary focus at Kickdex is on predictive models. We developed, together
with a colleague, an industry leading forecasting model which is used by a
multi-millions betting syndicate. Although technical details cannot be
disclosed, I can say that we built families of basic models using standard
machine learning classifiers (Random Forest, boosted trees, support vector
machine, neural networks, k-nearest neighbours, naive Bayes, \ldots{}) and other
models using more classic statistical techniques based on counting processes,
copula and survival analysis (for example, you can see this paper
\citep{boshnakov2017bivariate}). These basic models have been combined in an
\emph{ensemble} which is know to perform better than any single model taken
individually. We also leveraged techniques such as multi-task and transfer
learning. Besides, some effort has been spent on feature engineering, feature
selection, model performance testing \ldots{}

\item I also did some work on players evaluation. In this context, I created several
metrics under the \emph{REAL Analytics} (RA) label which is meant to help
football clubs use sounds mathematics to answer relevant questions for the game
of football:
\begin{itemize}
\item \emph{Plus-Minus Rating}: How important is the player for his team? The theory
has been published in a paper \citep{kharrat2017plusminus}. In particular,
we measure the player's importance in terms of goal differential (\texttt{PM}),
expected goal differential (\texttt{xGPM}) and expected points differential (\texttt{xPPM}).
\item \emph{valuing actions}: We developed an algorithm to compute the contribution of
every action in football to the probability of scoring/conceding goals. This
algorithm allow us to measure the importance of players' action and to
derive two objective ratings: \emph{performance} rating to answer the question how
did the player perform in a specific match? and \emph{overall} rating to answer
the question how good is the player right now (in terms of football skills)?
\item \emph{potential}: It is important to know how good a player will be in the future
leveraging some modern time series forecasting techniques. We developed an
algorithm to project the \emph{overall} rating in the future. Depending on the
player's age and his recent performance, we can estimate how good he is
likely to be.
\item \emph{reliability}: The RA reliability index estimates how often a player is
available to play. It takes into account information such as historical
injury and red card records and gives an estimate of the probability of a
player being available for selection of a given random match.
\item \emph{players' likely impact}: Together with a colleague at Salford business
school, we developed a set of algorithms labelled \emph{Sports Analytics Machine}
(SAM). The BBC is a prime user of SAM and among other things, SAM is able to
compute the impact of new signing in a team. In fact, including the new
signing in the squad and simulating the league path allow us to objectively
quantify the change of probability (with and without the player) of winning the
league, finishing in champions league positions \ldots{} An example of application
is given \href{https://www.bbc.co.uk/sport/football/37004327}{here}.
\end{itemize}
\end{itemize}
All these ratings and tools have been exposed in a \href{https://realanalytics.org}{web-app} I maintain (data
updated on regular basis). The different functionality are explained in this
\href{https://www.dropbox.com/s/k6ui3ugywcbcgtj/REAL\%20Analytics\%20demo.mp4?dl=0}{introductory-video}.

\begin{itemize}
\item \textbf{Tracking-data}: we have been mandated by a company (name cannot be disclosed)
to compare the quality of different tracking data (Prozone, inStat, TRACAB,
STATS). This project allowed us to familiarise ourselves with these new
generation of data and to prototype a new model to extend our \emph{valuing
actions} model which used only event-by-event ball data. However, the
confidentiality agreement and the small sample size didn't allow us to publish
our findings.
\end{itemize}

\section{Algorithmic Trading}
\label{sec:org108bc91}
\begin{itemize}
\item solid experience designing, back-testing and deploying trading strategies in
the US equity and football betting markets.
\item good experience working with high frequency (tick-by-tick, seconds, minutes
\ldots{}) price and volume data.
\item ability to discuss, explain and present trading results to (non technical)
stake holders audience.
\end{itemize}

\section{Consultancy}
\label{sec:org74f36f2}
Over the past 10 years, I have been involved in several consultancies as a team
leader or a technical expert in a specific subject:
\begin{itemize}
\item \emph{TOTAL (Oil and Gas Trading \& Shipping)}: tested and improved the STAGE
simulator, a software to simulate ships trips between loading and unloading
terminals taking into account real life constraints (weather, traffic,
dry-dock, \ldots{}).

\item \emph{major UK bookmaker}: auditing and improving the in-house forecasting models
for weak leagues and helping optimising the cash-out algorithm.

\item \emph{Atomic Weapons Establishment}: provided an algorithm to solve a dynamic
Poisson regression problem.

\item \emph{Thales}: created a software to model a gap in the coating of a submarine
\end{itemize}
\citep{heil2012quasi}.

\begin{itemize}
\item \emph{UEFA}: developed Aalgorithms to detect fixed games (joint work with sporting index).

\item \emph{Nottingham Forest F.C}: specific statistical reports on a list of target players.

\item \emph{BBC sports}: different applications of SAM (most likely score forecast, end
of season league table simulations, likely impact of a signing, players
importance \ldots{}).

\item \emph{major tracking data provider}: compare the accuracy of different tracking
data sources objectively (on going).

\item \emph{major UK law firm}: estimate the likelihood of the client (a footballer who
got injured in 2012) to make it to the top level in England (on going).
\end{itemize}
\section{Programming}
\label{sec:orgb93acdc}
\begin{itemize}
\item \textbf{Compiled languages}: strong knowledge of C++:
\begin{itemize}
\item contributed to the oomph lib library \citep{heil2006oomph}: author of the
Helmholtz module.
\item good working knowledge of the Armadillo \citep{sanderson2016armadillo} and
Eigen \citep{jacob2012eigen} libraries for linear algebra, NLopt
\citep{johnsonnlopt} for nonlinear optimisation and openMP
\citep{dagum1998openmp} for parallel computing.
\end{itemize}

\item \textbf{Interpreted languages}: 
\begin{itemize}
\item expert knowledge in \texttt{R}: author of several packages on \href{https://cran.r-project.org/}{CRAN}.
\item strong experience interfacing C++ code from R (to improve code performance).
\item ability to build comprehensive web apps using shiny and shiny dashboard.
\item basic understanding of python (learning it at the moment).
\item intermediate knowledge of Julia.
\end{itemize}

\item \textbf{Machine learning}: 
\begin{itemize}
\item solid understanding of machine learning algorithms for classification,
regression and clustering.
\item solid working experience with ML libraries such as h2o.ai
\citep{candel2016deep} and keras \citep{chollet2015keras}(R interface with
CPU and GPU backend).
\item extensive experience with the \texttt{caret} R package \citep{kuhn2008caret} for
model testing.
\item ability to deploy models in production environment leveraging new
technologies such as Ducker, version control (git) and conda manager.
\end{itemize}

\item \textbf{Database}:
\begin{itemize}
\item ability to build, maintain and store large amount of data using \href{https://www.mongodb.com/}{MongoDB} (no
sql) either locally or on the cloud (using \href{https://www.mongodb.com/cloud/atlas}{mongo-Atlas}).
\end{itemize}

\item \textbf{Parallel computing} 
\begin{itemize}
\item good experience with multithreading programming (using
\end{itemize}
the foreach package \citep{analytics2014doparallel} in R or openMP
\citep{dagum1998openmp} in C++) as well as GPU computing (using the RCUDA
package \citep{baines2014rcuda}).
\end{itemize}

\label{sec:orga89f5a4}
\bibliographystyle{apalike}
\bibliography{expertise}
\end{document}